\subsection{The Highly Scalable Data Storage Project}

% Intro paragraph describing the project aims
The HDF group in \dots
\todo{Update the description of the HSDS project}

We here repeat some of the requirements and assumptions from that project,
recognising that because we desire that the ESD is as compatible with the HSDS system as possible, most of these assumptions and requirements apply to the ESD
system as well.

\paragraph{Key Object Storage Assumptions}. The HSDS project makes the following assumptions about objects and the interactions with the object storage system:
\begin{enumerate}
    \item Files and Objects: Files are shared into objects.
    \item Key size: Object keys are limited to 1024 characters.
    \item Key names: The first 3-4 characters of the keys should be randomly distributed (to avoid request rate limits due to a single storage system be targeted for systems such as AWS where key names influence object location).
    \item Object size needs to follow the ``goldilocks principle'' needing to be not too small, and not too large: dealing with too many small objects would be prohibitively expensive in a public cloud and the use of very large objects ($>$ 100 MB) would introduce excessive latency.
    \item Updates to a storage object will be complete (i.e. the entire object will be overwritten), atomic (i.e. last writer wins), and either succeed or fail --- and in the latter case there should be no update to the object.
    \item Partial reads could be possible, though.
    \item The object storage system will not provide support for ``transactions'' (i.e. the ``all or nothing'' update of two or more objects).
    \item The storage system will not be read-write consistent.
    \item Metadata associated with each object will be at least 1024 bytes long.
    \item All objects managed by any service instance will exist in one ``bucket'' for which that service will need read-write authority.
\end{enumerate}

With these assumptions, we might expect that the HSDS (and hence ESD services) would need to perform a number of data management functions that are not provided by the underlying object storage paradigm:
\begin{enumerate} [resume]
    \item Listing of files, and file permissions, will need to be managed by a service (listing keys will generally be inefficient (and would not work well with randomly distributed keys), and file permissions are notions that belong in HSDS/ESD metadata space, not in the storage system).
    \item The reverse lookup between an object and the parent file will need to be explicilty managed by a service (an object will not itself know of its parent).
    \item All object creation and update needs to be managed through a service interface. Object retrieval may be either via a service which aggregates objects on the server side, or via a client which aggregates objects after interrogating a service to find appropriate objects.
\end{enumerate}
These functions are naturally provided by traditional (POSIX) file systems.
