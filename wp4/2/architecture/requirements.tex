\begin{chapterIntro}
The goal of this section is to provide high-level requirements: what the system needs to do and how it relates to dependencies. The chapter distinguishes between functional and non-functional requirements.
Functional requirements, that is the required features to fulfill the application of the system are enumerated in \Cref{sec: requirements/functional}.
Non-functional requirements that relate to the runtime qualities (e.g., performance, fault-tolerance or security) of the architecture are collected in \Cref{sec: requirements/non functional}.
\end{chapterIntro}

\label{chap:requirements}


\section{Functional Requirements}
\label{sec: requirements/functional}

The developed system is a storage system, thus provides basic means to access and manipulate data
and, thus, provides an API suitable for use in current and next generation high-performance simulation environments:

\begin{enumerate}
\item CRUD-operations -- Create, Retrieve, Update (append), Delete data in scientific relevant granularity.
\begin{itemize}
\item Partial access --
It must be possible to either retrieve (access) the complete results from experiments or to identify sections of interest and access those.
\end{itemize}
\item Discover, browse and list data. It must be possible to identify the file or object which contains interesting data,  and  eventually obtaining an identifier for the object and an endpoint through which it can be accessed.

\item Handling of scientific/structural metadata as first class citizen, that means the storage system understands the metadata and the API is designed to exploit this knowledge, e.g., data can be searched by consulting metadata catalogs.
\item Semantical namespace,  meaning that objects can be searched and accessed based on the structural metadata and not by a single hierarchical namespace.
\item Supporting heterogeneous storage -- the system shall exploit a heterogeneity of hardware technology, that means using the individual storage technologies for the best purpose, i.e., the characteristics of the storage define their use within ESD. At best, the system makes these decisions without user intervention but it may require users to provide certain ”hints” or intents how data is and will be used.

This includes cases such as:
\begin{enumerate}
    \item Caching data on faster storage tiers
    \item Explicit migration, where for example, users explicitly tag their data for a ``lower'' tier of storage (cheap and/or slow), but the ESD system needs to cache the data en-route to tape.
    \item Overflow, where for example a particular deployed ESD system is unable to handle new data stores to disk without flushing old data to tape.
    \item Transparent (and/or non-transparent) data migration, e.g., data migrates from tape to disk in response to full or partial read requests through one of the ESD interfaces.
\end {enumerate}


%Currently most weather and climate centres deploy an HPSS system of some sort in their environment, which is usually very costly. The ESD storage system will need to have at least some sort of internal hierarchy and caching system.

\item Function shipping -- support the transfer of compute kernels to the storage system and process data somewhere in the I/O path. This reduces data movement which is costly on Exascale systems.

\item Compatibility -- for backwards compatibility with existing climate and NWP applications, the system must expose or support existing APIs, e.g.,
\begin{itemize}
\item a NetCDF interface
\item an HDF5 interface %suitable to be bound into client HDF libraries deployed both locally (LAN) and remotely (WAN),
\item a GridFTP interface
\item a POSIX file system interface
\item a suitable RESTful interface
\end{itemize}
In particular, it shall be possible to create data using one interface and accessing the data without conversion using another.
\end{enumerate}



These mandatory requirements are accompanied by supporting requirements:
\begin{enumerate}
\item Auditability -- upon request, object-specific operations need to be logged, in particular, all creations, retrievals, and updates discriminated by users.
\item Configurability -- A system wide configuration of all available storage resources and their performance characteristics must be possible.
\item Notifications -- A tool or user may subscribe for a object and be notified if certain modifications are made to the object. This allows to watch the changelog of objects efficiently without polling.
\item Import/Export -- tools to support data exchange in or out of the ESD system.
Depending on the format, conventions for mapping the format internal metadata or supplying metadata needed to meet  internal ESD metadata requirements.
\item Access control -- it should be able to restrict access to objects, supporting several roles, e.g., data center staff, users, projects.
\begin{itemize}
\item Data sharing -- In particular, the system should make it simple to share data with other researchers.
\end{itemize}
\end{enumerate}



\section{Non-Functional Requirements}
\label{sec: requirements/non functional}


For a productive use in a data centre, these requirements are mandatory:
\begin{enumerate}
\item Performant --- coping with volume/velocity and delivering adequate bandwidth and latency. This also means the system is efficient and exploits available performance from hardware (network and storage technology), at best 90\%+ of the available raw network bandwidth and storage system throughput is achievable from applications.
\item Reliable --- stored data is durable. In transit data corruption and silent data corruption is detected and corrected. In case an uncorrectable error occurs, the data that is affected can be identified and the still correct parts of affected objects shall still be accessible. Reliability includes:
  \begin{itemize}
  \item Fault Tolerant --- able to detect and repair hardware and software failures.
  \item Highly-available --- data access shall be possible even with partially broken hardware and software, thus, meeting expectations of users similar to that they have of conventional disk sub-systems.
  \end{itemize}
\item Versatile ---  able to utilise existing heterogeneous storage infrastructures, including, but not limited to
  \begin{itemize}
  \item existing traditional storage systems such as tape, object store, and parallel file system,
  \item existing non-traditional storage systems such as the HDFS, and
  \item potential new storage systems such as burst buffers or NVRAM.
  \end{itemize}
\item User-friendly -- the system shall provide a good usability, this includes:
  \begin{itemize}
  \item Transparency --- hides specifics of the storage landscape and does not \textit{\textbf{require}} users to set and define technical parameters specifically to a given system.
  \item Portability --- code that uses the developed APIs should work in different data centres, i.e., with different hardware and software.
  \end{itemize}
\item Cost-Effective --- the software system should be affordable and maintainable by data centre staff at Exascale.

\item Standards based --- using interfaces, formats and standards which maximise re-usability.
\end{enumerate}
