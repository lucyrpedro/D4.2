

%\begin{longtable}{| >{\centering\arraybackslash} m{5.5cm} | >{\centering\arraybackslash} m{6cm} |}\hline\hline
%        \cellHeader Datatype & \cellHeader Description \\ \hline
%        \centering \small ESD\_TYPE\_CHAR & \small 8-bit character \\ \hline
%        \caption{ESD Datatypes}
%        \label{table: esd-type}
%\end{longtable}

%\begin{longtable}{| >{\centering\arraybackslash} m{5.5cm} | >{\centering\arraybackslash} m{6cm} |}\hline\hline
%        \cellHeader Type Constructor & \cellHeader Description \\ \hline
%        \centering \small - & \small - \\ \hline
%        \caption{ESD Datatype Constructors}
%        \label{table: esd-constr}
%\end{longtable}

%\todo{What type of datatypes are supported? In principle the ESD can support NetCDF and HDF5 by providing the same datatypes as for HDF5. Are these enough? Need to be extended to support some other type?}.



%%%%%%%%%%%%%%%%%%%%%%%%%%%%%%%%%%%%%%%%%%%%%%%%%%%%%%%%%%%%%%%%%%%%%%%%%%%%%%%
\subsubsection{Adaptive Data Mappings}
The typical storage mapping for scientific data format libraries is the file (linear sequence of bytes organized following the POSIX file system representation, i.e. inodes and blocks). Information is translated into a linear array of bytes in the file system using appropriate schemas. Since the data model defined by the data format library can contain complex hierarchies and attributes besides raw data, the final file will contain additional scientific metadata that needs to be accessed using the POSIX-IO data interface (e.g. \texttt{read()} and \texttt{write()}) instead of the file metadata interface (e.g. \texttt{stat()}, \texttt{lookup()}). This makes the storage model adopted by data format libraries incompatible with the typical parallel file system organization, in which metadata and data are splitted apart and assigned to different services for optimal performance. Additionally, new storage system paradigms have emerged in the last years in which files are organized in a flat namespace (e.g. object storage), removing the restrictions imposed by metadata operations like namespace traversal of POSIX file systems. Hierarchical organization can be still achieved using other dedicated storage representations like key-value stores, at the expense of reduced POSIX semantic.

There is therefore a necessity for more flexible data mappings that can take advantage of an increasing number of storage and backend alternatives, improving access efficiency at the same time. There are two different approaches to this problem: the first is to develop a new data mapping schema for every storage backend. This is the solution adopted by the HDF5 library (through the virtual file and the virtual object layers) in which multiple storage backends can be employed by developing a corresponding plugin that contains the right mapping schema. The obvious limitation of this solution is that once the user has selected a certain backend for a file, this cannot be changed without migrating the whole file to another backend. The second approach is more flexible and consists in making the storage backend selection adaptive. This has the advantage of enabling backend storage selection on the fly depending upon the type of data being stored or a set of user/system defined parameters (e.g. list of data placement policies satisfying certain requirements of quality of service).

Our proposed solution is to integrate the adaptive data mappings capabilities just discussed in the layout component of the ESD middleware. The middleware will understand the scientific metadata of other relevant data format libraries and will be able to use multiple storage backends at the same time to store and retrieve different pieces of data. To support multiple formats the datatype component in the ESD middleware will expose a datatype interface similar to the one available in HDF5 (described in Section~\ref{sec: data-formats}). Additionally, the middleware will add significant metadata to be used in the life cycle management and sharing of data. This can include semantic metadata describing which data is needed and how it is going to be used, the required level of resilience, etc. In this context, some metadata can be automatically generated by the middleware or defined by the user and passed to the middleware through an apposite interface (that will be part of the ESD exposed interface).

Legacy codes will have access to data using a familiar POSIX interface exported through an ESD FUSE module. The FUSE module will communicate with the ESD middleware to access data and export it to users using a namespace representation. The mapping schema definition for mapping data between the storage representation and the namespace will be done later in the project.

Another important aspect to consider when talking about adaptive data mappings is the storage tiering, that is, how many levels of storage there are in the system. Historically, HPC applications have relied on parallel file systems as first tier to store and retrieve their data. Besides the parallel file system, most high-end clusters have a second level of archival storage to which data is migrated from the parallel file system using a hierarchical storage manager. Nowadays compute nodes have access to local storage (typically a block device formatted with a local file system) and with the emergence of new storage technologies, such as non-volatile memories, permanent byte addressable memory. The ESD middleware should be able to exploit these local storage resources to implement prefetching strategies (read patterns) and burst buffers (write patterns).

%In the specific case of Mero (Seagate object store system), which represents one of the possible storage backends that has to be supported by the ESD middleware, hierarchical and semantic metadata can be stored in the key-value store service while array data can be stored into objects. In fact, we can think about using more than one object per dataset. This is particularly useful with chunking since every chunk in the dataset can be mapped to a different object, making parallel access to different data units possible without any need for I/O coordination (no false sharing is possible in this case). Of course, I/O coordination still needs to be provided when multiple processes access the same dataset and chunking is not enabled. In this case the ROMIO library already provides a collective I/O implementation that can be reused by the ESD middleware. Whenever needed, false sharing of data can also be avoided by using `persistent file realm' like mechanisms\footnote{This mechanism allows a certain file portion to be assigned to a specific process in the application and makes this assignment persistent across multiple data accesses.}.

%\todo{Giuseppe: you had a very good document on this already can add something here}

%Goal of the dynamic mapping is to create a hierarchical namespace based on the available metadata.
%It is based on the search for objects.
%With the help of FUSE, these mappings can be created based on a user configuration upon mount-time (supports user mounts).
%Inaccessible data, i.e., where the permissions are not sufficient, can be hidden from the namespace.

%An example mapping could be:
%“v.info.model/v.info.environment.date/v.info.experiment.tags/v.info.name <as NC4>”.
%This would show a single NetCDF4 file for each variable.
%If a variable is attached to various tags, then it is shown for each of them.

%“v.info.model/v.info.environment.date/v.domain.time/v.info.name <as NC4>”.
%Would show one file per timestep.

%How to resolve ambiguity?

%Existing containers can also be directly mapped, showing the input and output variables similarly:
%“<c.\_id>/<c.directory>”


%The storage should now how the logical data space maps to physical positions in memory or storage.
%If the storage backends have this knowlege, they can run local operations on the data.

%Alternatives:
%Triangular grid (locally refined), somehow mapped from user-space to a 1D structure which gets mapped to the storage. (Traditional approach)

%Application(data structure) - Mapping - Middleware - Mapping (byte array) - Storage API - Storage backend

%or:

%Application - Middleware - Storage API - Mapping - Storage backend

%Use sth. like MPI datatypes to descripe memory (in the application)
%Add metadata to describe the meaning / semantics, additional gain: resilience because we know which data to replicate to put into triple ECC ...
%Add hints how the data is used / will the used in the future

%Avoid redundant descriptions of memory / storage.
%Users stores a 3D field, it is clear which storage location is responsible for [x,y,z]
%How to store triangles etc.

%Why not store code as description?
%- describes neighborhood?

%Coordinates? Universal coordinates, application coordinates?

%VTK? haben diverse Datenstrukturen hierfÃŒr.


%Separation of concerns...

%Missmatch of Chunks size and application domain

%false sharing

%Pointer Datentyp: strong ptr, weak ptr.
%like Boost usw.


%NVRAM
